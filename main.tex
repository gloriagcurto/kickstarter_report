%\documentclass[a4paper]{article}

\documentclass{article}
\usepackage[a4paper]{geometry}

\usepackage[utf8]{inputenc}
\usepackage{import}
\usepackage{url}
\usepackage{graphicx}
\usepackage{float} 
\usepackage{xcolor}
\usepackage{subfigure}
\usepackage{caption}
\usepackage{listings} % JSON listing
\usepackage{eurosym}
\usepackage{parskip} % Do not indent the beginning of each paragraph
\usepackage{comment}

%\renewcommand{\sectionname}{Lecture} % Redefine "chapter" to a different text

\newcommand{\ToDo}[1]{\textcolor{magenta}{\textbf{[ToDo]} \textbf{#1}}}
\newcommand{\miguel}[1]{\textcolor{blue}{[Miguel] {#1}}}


\title{Classification of Kickstarter's successful or failed projects based on data crawled by a scraper robot (Web Robots)}
\author{Gloria GONZÁLEZ CURTO}
\date{\today}

\begin{document}

\maketitle

\tableofcontents


\section{Summary}
\label{ref:sum}
The aim of this project is to classify and ultimately predict the success or failure of a kickstarter fund rising campaign based on data monthly crawled from the kickstarter.com web site by a scraper robot (webrobots.io).

\section{Problem definition and background}
{Kickstarter is an American corporation with a public benefit status launched in 2009 in the United States. The company initiated its international expansion in 2012. The kickstarter platform is open to backers from anywhere in the world and to creators from many countries \ref{subsec:EDA}.

Kickstarter maintains a global crowdfunding platform focused on creativity. Project creators choose a deadline and a minimum funding goal. If the goal is not met by the deadline, no funds are collected and the project is considered failed. Project backers usually receive rewards in exchange for their pledges.
Once a project collects enough pledges to bypass the goal before a deadline stablished by the creator, its state changes to successful. Only at that moment pledges are collected from backers. From that point onward, there is no guarantee for project delivery. Kickstarter advises backers to use their own judgment on supporting a project. They also warn project leaders that they could be liable for legal damages from backers for failure to deliver on promises.

From a backer point of view, it would be interesting to have an analytic model to help with the decision of whether to pledge for a project or not.

\section{Data description and pre-processing}
\label{sec:data_desc}

Raw data were collected from \url{https://webrobots.io/kickstarter-datasets/} in CSV format.  The downloaded data correspond to a crawl executed on February 13 2020. Raw data are composed by 57 CSV files with 3500 to 4000 rows each. The raw original variables are:
\begin{itemize}

    \item \textbf{backers\_count}: integer representing how many people supported the project (backers).
    \item \textbf{blurb}: textual description of the project (around 150-200 characters).
    \item \textbf{category}: JSON-encoded description on the kickstarter's fixed categories: id, name, slug, position, parent\_id, parent\_name, color, urls.

    \item \textbf{converted\_pledged\_amount}: pleged amount converted to USD.
    \item \textbf{country}: country initials (i.e: FR for France):
    \item \textbf{country\_displayable\_name}: country name.
    \item \textbf{created\_at}: date on Unix time format.
    \item \textbf{creator}: JSON-encoded description on project's creator. id, name, is\_registered, chosen\_currency, is\_superbacker, avatar, small, medium, url\_web, url\_api. 
    \item \textbf{currency}: currency acronym.
    \item \textbf{currency\_symbol}.
    \item \textbf{currency\_trailing\_code}: boolean variable.
    \item \textbf{current\_currency}: USD.
    \item \textbf{deadline}: crowdfunding deadline in Unix time established by the project's creator.
    \item \textbf{disable\_communication}: boolean variable.
    \item \textbf{friends}.
    \item \textbf{fx\_rate}: conversion rate.
    \item \textbf{goal}: fund rising goal in creator's fixed currency.
    \item \textbf{id}: project identifier.
    \item \textbf{is\_backing}.
    \item \textbf{is\_starrable}.
    \item \textbf{is\_starred}.
    \item \textbf{launched\_at}: launch date on Unix time format.
    \item \textbf{location}: JSON-encoded description of the project's creator location: id, name, slug, short\_name, displayable\_name, localized\_name, country, state, type, is\_root, expanded\_country, url\_web, url\_location, url\_api\_nearby\_projects.
    \item \textbf{name}: project's name.
    \item \textbf{permissions}.
    \item \textbf{photo}: JSON-encoded description on project's photos: key, full, ed, med, little, small, thumb, 1024x576, 1536x864.
    \item \textbf{pledged}: pledged amount in original country currency.
    \item \textbf{profile}: JSON-encoded description on project's profile: id, project\_id, state, state\_changed\_at, name, blurb, background\_color, text\_color, link\_background\_color, link\_text\_color, link\_text, link\_url, show\_feature\_image, background\_image\_opacity, should\_show\_feature\_image\_section, feature\_image\_attributes, image\_urls. 
    \item \textbf{slug}: small textual description with - as spacers.
    \item \textbf{source\_url}: url pointing to project's category site on kickstarter.
    \item \textbf{spotlight}: boolean representing if the project was featured on kickstarter web site.
    \item \textbf{staff\_pick}: boolean representing if the project was picked by kickstarter staff.
    \item \textbf{state}: project's state initialy composed by 5 categories (successful, failed, canceled, suspended and live.
    \item \textbf{state\_changed\_at}: date in Unix time format.
    \item \textbf{static\_usd\_rate}: static rate for currency conversion into USD.
    \item \textbf{urls}: JSON-encoded description containing urls: project, rewards.
    \item \textbf{usd\_pledged}.
    \item \textbf{usd\_type}: domestic.
\end{itemize} 

\subsection{Data pre-processing}
\label{subsec:data_prepro}
\subsection{Exploratory data analysis}
\label{subsec:EDA}
\section{Data modeling}
\label{sec:data_model}
\section{Model interpretation}
\label{sec:interp}
\section{Conclusions}
\label{sec:conclu}

\begin{comment}
\begin{figure}[!ht]
\centering
\includegraphics[width=0.8\linewidth]{RRPR_2016/images/architecture/architecture.pdf}
\caption{IPOL as a modular system.} 
\label{fig:architecture}
\end{figure}
\end{comment}

\begin{comment}
\begin{itemize}
    \item Demos
    \item Applications
\end{itemize}
\end{comment}

\bibliographystyle{unsrt} % Sorted references: [1], [2], ..
\bibliography{bibliography}

\end{document}
